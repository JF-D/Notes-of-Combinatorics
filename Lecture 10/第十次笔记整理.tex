\documentclass[11pt]{article}

\usepackage[a4paper]{geometry}
\geometry{left=2.0cm,right=2.0cm,top=2.5cm,bottom=2.5cm}

\usepackage{ctex}
\usepackage{amsmath,amsfonts,graphicx,amssymb,bm,amsthm}
\usepackage{algorithm,algorithmicx}
\usepackage[noend]{algpseudocode}
\usepackage{fancyhdr}
\usepackage{hyperref}

\newcounter{counter_exm}\setcounter{counter_exm}{1}
%\newcounter{counter_thm}\setcounter{counter_thm}{1}
%\newcounter{counter_lma}\setcounter{counter_lma}{1}
%\newcounter{counter_dft}\setcounter{counter_dft}{1}
%\newcounter{counter_clm}\setcounter{counter_clm}{1}
%\newcounter{counter_cly}\setcounter{counter_cly}{1}

\newtheorem{theorem}{{\hskip 1.7em \bf 定理}}
\newtheorem{lemma}[theorem]{\hskip 1.7em 引理}
\newtheorem{proposition}[theorem]{Proposition}
\newtheorem{claim}[theorem]{\hskip 1.7em 命题}
\newtheorem{corollary}[theorem]{\hskip 1.7em 推论}
\newtheorem{definition}[theorem]{\hskip 1.7em 定义}

\renewcommand{\emph}[1]{\begin{kaishu}#1\end{kaishu}}

\newenvironment{solution}{{\noindent\hskip 2em \bf 解 \quad}}


\renewenvironment{proof}{{\noindent\hskip 2em \bf 证明 \quad}}{\hfill$\qed$\par}
\newenvironment{example}{{\noindent\hskip 2em \bf 例 \arabic{counter_exm}\quad}}{\addtocounter{counter_exm}{1}\par}

\newenvironment{concept}[1]{{\bf #1\quad} \begin{kaishu}} {\end{kaishu}\par}

\newcommand\E{\mathbb{E}}

\begin{document}

    \pagestyle{fancy}
    \lhead{\kaishu 中国科学院大学}
    \chead{}
    \rhead{\kaishu 2017年秋季学期组合数学}

    \begin{center}
        {\LARGE \bf 组合数学第十讲}\\
    \end{center}
        \begin{kaishu}
            授课时间: 2017年11月21日\quad
            授课教师: 孙晓明
            \hfill 记录人: 姜茜\quad 王语
        \end{kaishu}

    \section{思考题:证明斯特林数为单峰序列}
    根据李奉治同学给出的解答,现将此整理至笔记中,供同学们学习参考。

    求证:$S_1(n,k)$与$S_2(n,k)$是单峰数列。即对于任何自然数$n \leq 3$,存在自然数$t_1,t_2 \leq n$满足:
    \begin{align*}
        S_1(n,0) \leq \cdots \leq S_1(n,t_1-1)<S_1(n,t_1)>S_1(n,t_1+1) \geq \cdots \geq S_1(n,n)\\
        S_2(n,0) \leq \cdots \leq S_2(n,t_2-1)<S_2(n,t_2)>S_2(n,t_2+1) \geq \cdots \geq S_2(n,n)\\
    \end{align*}
    \begin{definition}[对数凹性]
        正数数列$a_0,a_1,a_2,\cdots,a_n$具有对数凹性,当且仅当对于任意正整数$1\leq k\leq n-1$,满足$a^{2}_k \geq a_{k-1}a_{k+1}$。
    \end{definition}

    \begin{lemma}
        具有对数凹性的数列是单峰数列。
    \end{lemma}
    \begin{proof}
        对于数列$a_0,a_1,a_2,\cdots,a_n$,若其满足对数凹性,则满足$a^{2}_k \geq a_{k-1}a_{k+1}$,故$\frac{a_{k+1}}{a_{k}} \leq \frac{a_{k}}{a_{k-1}}$。由此可得出,若$a_{k} \leq a_{k-1}$,则$a_{k+1} \leq a_{k}$。因此数列$\{ a_n \}$会递增直至满足$a_{t} \leq a_{t-1}(t \geq 1)$,满足单峰数列条件。
    \end{proof}

    \begin{lemma}
        若一个正数数列对应的生成函数只有负实根,则这个数列满足对数凹性。
    \end{lemma}
    \begin{proof}
        对于正数数列$a_0,a_1,a_2,\cdots,a_n$,其生成函数为$P_n(x)=\sum^{n}_{k=0}a_{k}x^{k}$。使用数学归纳法证明。当$n=2$时,$P(x)=a_0+a_1x+a_2x^2$,由于系数均为正数,易证其无非负实根。我们仅需证明其根均为实根时,数列满足对数凹性;又根据判别公式,方程的根均为实根,需要满足$\Delta = b^2-4ac = a_1^2-4a_0a_2 \geq 0$,移项可得$a_1^2 \geq 4a_0a_2 \geq a_0a_2$,这说明数列满足对数凹性。

        归纳假设当$n=t-1$时原命题成立,即:若$P(x)=\sum^{t-1}_{k=0}a_{k}x^{k}$只有负实根,则正数数列$\{a_k\}_{k=0}^{t-1}$满足对数凹性。当$n=t$时,若长度为$t+1$的正数数列$\{a_k\}_{k=0}^t$的生成函数$P(x)=\sum^{t}_{k=0}a_{k}x^{k}$只有负实根,那么其必然可以表示为$P(x)=(x+c)Q(x)$,其中$c>0$且$Q(x)=\sum^{t-1}_{k=0}b_{k}x^{k}$,易知数列$\{a_k\}_{k=0}^t$可以表示为:
        \[
            a_k=\left\{\begin{array}{cl}
                cb_0 &\textrm{if $k=0$;}\\
                b_{t-1} &\textrm{if $k=t$;}\\
                b_{k-1}+cb_k &\textrm{otherwise.}
            \end{array}\right.
        \]
        根据假设$Q(x)$只有负实根,且根据$n=t-1$时的假设,有$b_k^2 \geq b_{k-1}b_{k+1}$。那么当$2 \leq k \leq n-2$时:
        \begin{gather*}
            b_k^2b_{k-1} \geq b_{k+1}b_{k-1}^2 \geq b_{k+1}b_kb_{k-2}\\
            b_kb_{k-1} \geq b_{k+1}b_{k-2}
        \end{gather*}
        即可得到:
        \begin{align*}
            a_k^2-a_{k+1}a_{k-1}&=(b_{k-1}+cb_k)^2-(b_k+cb_{k+1})(b_{k-2}+cb_{k-1})\\
                &=(b_{k-1}^2-b_kb_{k-2})+c(b_{k-1}b_k-b_{k+1}b_{k-2})+c^2(b_k^2-b_{k+1}b_{k-1})\\
                &\geq 0
        \end{align*}
        当$k=1$时:
        \begin{align*}
            a_1^2-a_2a_0&=(b_0+cb_1)^2-(b_1+cb_2)cb_0\\
                &=b_0^2+cb_0b_1+c^2b_1^2-c^2b_0b_2\\
                &\geq b_0^2+cb_0b_1+c^2b_0b_2-c^2b_0b_2\\
                &\geq 0
        \end{align*}
        当$k=n-1$时:
        \begin{align*}
            a_{n-1}^2-a_na_{n-2}&=(b_{n-2}+cb_{n-1})^2-b_{n-1}(b_{n-3}+cb_{n-2})\\
                &=b_{n-2}^2+cb_{n-1}b_{n-2}+c^2b_{n-1}^2-b_{n-1}b_{n-3}\\
                &\geq b_{n-1}b_{n-3}+cb_{n-1}b_{n-2}+c^2b_{n-1}^2-b_{n-1}b_{n-3}\\
                &\geq 0
        \end{align*}
        综上可得,$a_k^2 \geq a_{k+1}a_{k-1}$,归纳成立,引理得证。
    \end{proof}

    第一类斯特林数的母函数为:
    \[
        \sum_{k=1}^{n}S_1(n,k)x^k=x(x+1)(x+2)\cdots(x+n-1)
    \]
    即第一类斯特林数的生成函数有根$0,-1,-2,\cdots,-(n-1)$。为消除零根,可不考虑$S_1(n,0)$,等式左右两边同除$x$,左侧相当于系数平移,右侧去除了零根。这时,根据上述引理可得,第一类斯特林数的数列为单峰数列。

    第二类斯特林数的有如下递推:
    \[
        S_2(n,k)=S_2(n-1,k-1)+kS_2(n-1,k)\qquad\forall1<k<n
    \]
    使用数学归纳法,证明第二类斯特林数的生成函数只有零根和负实根,然后用和处理第一类斯特林数问题相同的方法去除零根,证明最终结果。

    记$P_n(x)=\sum_{k=0}^{n}S_2(n,k)x^k$。由于$S_2(n,0)=0,\forall n>0$,因此$0$一定为$P_n(x)$的一个根。当$n=1$时,$P_1(x)=x$只有一个零点$x=0$;当$n=2$时,$p_2(x)=x+x^2$有两个零点$x_1=0$和$x_2=-1$;当$n>2$时,根据递归式,有:
    \begin{align*}
        P_n(x)&=\sum_{k=1}^nS(n,k)x^k\\
            &=\sum_{k=1}^nS(n-1,k-1)x^k+\sum_{k=1}^nkS(n-1,k)x^k\\
            &=xP_{n-1}(x)+x\frac{dP_{n-1}(x)}{dx}
    \end{align*}
    令$Q_n(x)=P_n(x)e^x$,易知$P_n(x)=0$和$Q_n(x)=0$有相同的根.将上述等式左右同乘$e^x$可以得到
    \[
        Q_n(x)=x\frac{dQ_{n-1}(x)}{dx}
    \]
    通过递推条件,已知$Q_{n-1}(x)$有零根和$n-2$个负实根,且易推出$${\lim_{x \to -\infty}Q_{n-1}(x)=0}$$
    根据罗尔中值定理,上述$n$个点间存在$n-1$个导数为$0$的点,这些点都为负实数。再加上零根,正好构成$Q_n(x)$的$n$个根。综上,第二类斯特林数的生成函数只有零根和负实根,原命题得证。

    同学们可以尝试证明分拆数$P_{n,k}$也是单峰数列。

    \section{伯兰特-切比雪夫定理与素数定理}

    \subsection{证明$2^{32}+1$不是素数}
    上次课提到欧拉证明了$2^{32}+1$不是素数,这里给出证明:

    \begin{proof}
        注意到$641=5^4+2^4=2^7\cdot 5+1$,所以有:
        \begin{align*}
            2^{32}+1&=(2^7)^4\cdot16+1=(2^7)^4\cdot(641-625)+1\\
            &=2^{28}\cdot641+1-(2^7\cdot5)^4=2^{28}\cdot 641+(1+640)\cdot(1-640)\cdot(1+640^2)\\
            &=641\cdot (2^{28}-639\cdot(1+640^2))
        \end{align*}
        由此得出,$2^{32}+1$含有因子641,不是素数。
    \end{proof}

    \subsection{伯兰特-切比雪夫定理(Bertrand–Chebyshev theorem)及其证明}
    \begin{theorem}
       对于任意大于等于$3$的正整数$n$,在$(n,2n]$中存在素数。
    \end{theorem}
    \begin{proof}
        上次课中,我们给出了证明的前半部分。我们采用反证法,假设在$(n,2n]$之间不存在素数。依据这一假设,我们将从$\binom{2n}{n}$出发,试图给出$\binom{2n}{n}$的一个上界,并说明当$n$充分大时这个上界的量级要小于其实际下界的量级从而引出矛盾。
        上次课中,我们证明了
         \[
            \binom{2n}{n}=\prod_{p\in P(2,\sqrt{2n})}p^{h_p(2n)-2h_p(n)}\cdot \prod_{p\in P(\sqrt{2n}+1,\frac{2n}{3})}p^{\lfloor\frac{2n}{p}\rfloor-2\lfloor\frac{n}{p}\rfloor}
         \]
        其中$P(n,m)$表示$[n,m]$区间的素数集合。$\prod_{p\in P(2,\sqrt{2n})}p^{h_p(2n)-2h_p(n)}$的量级已经在上节课给出,余下的证明将给出$\prod_{p\in P(\sqrt{2n}+1,\frac{2n}{3})}p^{\lfloor\frac{2n}{p}\rfloor-2\lfloor\frac{n}{p}\rfloor}$的一个上界。

        对于任意自然数$m$有:
        \[
            \prod_{p\in P(2,m)}p=\prod_{p\in P(2,\lfloor\frac{m}{2}\rfloor)}p \cdot \prod_{p\in P(\lfloor\frac{m}{2}\rfloor+1,m)}p
        \]
        因为$m-\lfloor\frac{m}{2}\rfloor=\lceil\frac{m}{2}\rceil$,考虑:
        \[
            \binom{m}{\lfloor\frac{m}{2}\rfloor}=\frac{m!}{\lfloor\frac{m}{2}\rfloor!\lceil\frac{m}{2}\rceil!}=\frac{m\cdot(m-1)\cdots(\lceil\frac{m}{2}\rceil+1)}{\lfloor\frac{m}{2}\rfloor!}
        \]
        当$m$是偶数,即$m=2k$时,$\binom{2k}{k}=\frac{2k\cdot(2k-1)\cdots(k+1)}{k!}$,对于大于$k$的素数,它们都出现在分子中,但都不会出现在分母中,所以不会被分母中的数约掉,所以有$\left(\prod_{p\in P(k+1,2k)}p\right)\mid \binom{2k}{k}$。当$m$是奇数,即$m=2k+1$时,对于大于$k$的素数,除去$k+1$(如果它是素数),都出现在分子中但不在分母中,所以有$\left(\prod_{p\in P(k+1,2k+1)}p\right)\mid (k+1)\binom{2k+1}{k}$。综上所述:
        \[\left(\prod_{p\in P(\lfloor\frac{m}{2}\rfloor+1,m)}p\right)\leq \left\lceil\frac{m}{2}\right\rceil\binom{m}{\lfloor\frac{m}{2}\rfloor}\]
        设$P(m)=\prod_{p\leq m}p$,由此得到递推:
        \[\prod_{p\in P(2,m)}p\leq \left(\prod_{p\in P(2,\lfloor\frac{m}{2}\rfloor)}p\right)\left\lceil\frac{m}{2}\right\rceil\binom{m}{\lfloor\frac{m}{2}\rfloor}\]
        解得:
        \[
            \prod_{p\in P(2,m)}p\leq 2^{2m+(\log_2 m)^2}
        \]
        最终得到:
        \[
            \Omega\left(\frac{4^n}{2n}\right)=\binom{2n}{n}=\tilde O\left((2n)^{\frac{\sqrt{2n}}{2}}\cdot {4}^{\frac{2n}{3}}\right)
        \]
        显然左侧的增长要远快于右侧,当$n$充分大时引发矛盾,原命题得证。
      \end{proof}

    \subsection{素数定理(Prime number theorem)及其简单版本的证明}
    \subsubsection{素数定理叙述}
    素数定理描述素数的大致分布情况,定义
    \[\pi(n):=|\{p|p\leq n,p\textrm{是素数}\}|\]
    素数定理就是尽可能地逼近$\pi(n)$,可以证明:
    \[\pi(n)\sim\frac{x}{\ln x}\]
    即:
    \[\lim_{x\to\infty} \frac{\pi(x)}{x/\ln x} = 1 \tag{1}\]

    1792年,卡尔·弗里德里希·高斯(Carl Friedrich Gauss)提出素数定理,几乎在同一时候,阿德里安-马里·勒让德(Adrien-Marie Legendre)猜想存在上述的逼近公式。巴夫尼提·切比雪夫(Pafnuty Chebyshev)对素数定理的研究作出了重要贡献,他证明:存在两个正实数$C_1$和$C_2$,使如下不等式对充分大的$x$成立:
    \[C_1\frac{x}{\ln x} \leq \pi(x) \leq C_2\frac{x}{\ln x}\]
    其中$C_1=0.92...,C_2=1.055...$。

    素数定理最终由雅克·阿达马(Jacques Hadamard)和查尔斯·德拉瓦莱普森(Charles de la Vallée-Poussin)在1896年分别独立给出证明。1948年,阿特勒·塞尔伯格(Atle Selberg)和保罗·埃尔德什(Paul Erd\"os)给出了素数定理的初等证明,只需用数论方法证明。有必要指出的是,虽然该初等证明只用到初等的办法,却比用到复分析的证明更为复杂。

    \subsubsection{素数定理的部分证明}
    由2.2节的证明得知:
    \[\prod_{p \in P(2,n)}p \leq 2^{2n+(\log_{2}n)^2}\]
    又注意到:
    \[\prod_{p \in P(2,n)}p \geq \prod_{p\in P(\frac{n}{2}+1,n)}p \geq \left(\frac{n}{2}\right)^{\pi(n)-\pi(\frac{n}{2})}\]
    得到如下递推式:
    \[\pi(n) \leq \pi\left(\frac{n}{2}\right)+O\left(\frac{n}{\ln n}\right)\]
    不妨设存在$k$令$\sqrt{n}=\frac{n}{2^k}$,解得:
    \begin{align*}
        \pi(n) &\leq O\left(\frac{n}{\ln n}+\frac{\frac{n}{2}}{\ln \frac{n}{2}}+\cdots+\frac{\sqrt{n}}{\ln \sqrt{n}}+\frac{\frac{\sqrt{n}}{2}}{\ln {\frac{\sqrt{n}}{2}}}+
            \frac{\frac{\sqrt{n}}{4}}{\ln {\frac{\sqrt{n}}{4}}}+\cdots\right)\\
        \pi(n) &\leq O\left(\frac{n}{\ln \sqrt{n}}+\frac{\frac{n}{2}}{\ln \sqrt{n}}+\cdots+\frac{\sqrt{n}}{\ln \sqrt{n}}+\frac{\frac{\sqrt{n}}{2}}{1}+
            \frac{\frac{\sqrt{n}}{4}}{1}+\cdots\right)\\
        \pi(n) &\leq O\left(\frac{4n}{\ln n}+O(\sqrt{n})\right)\\
        \pi(n) &= O\left(\frac{n}{\ln n}\right)
    \end{align*}
    上面的证明放缩得比较宽松,如果将$\frac{n}{2}$换成$\sqrt{n}$,$n^{\varepsilon}$等,可以将$C_{1}$,$C_{2}$进一步逼近。但是,这不足以证明素数定理。

    $\pi(n)$下界的证明留作作业。

    \section{狄利克雷定理(Dirichlet's theorem)}
        \begin{theorem}
            对于任意互素的正整数$a,b$,存在无穷多个自然数$k$,满足$ak+b$是素数。
        \end{theorem}
        该定理由勒热纳·狄利克雷(Lejeune Dirichlet)于1837年证明,从另一个侧面提供了素数分布的刻画。本节我们将证明两个狄利克雷定理的特例。
    \subsection{$4k+3$型素数无穷}
    反证假设$4k+3$型的素数只有有限个,记为$p_{1},p_{2},...,p_{n}$。考虑素数
    \[N=4p_{1}p_2\cdots p_{n}-1 \equiv 3 \pmod4\]
    显然,$p_{1},...,p_{n}$不是$N$的素因子。除了素数2,其余素数均不为偶数。所以$N$的所有素因子都是$4k+1$型,因为$4k+1$型的素数乘积依然$4k+1$型的,即$N\equiv 1\pmod 4$,矛盾!

    按照相同的思路,我们还可以证明模$6$余$5$的素数有无限多个。
    \subsection{$4k+1$型素数无穷}
    首先,引入二次剩余的概念。

    \begin{definition}[二次剩余]
        对于正整数$a,n$,当存在整数$b$满足$b^2\equiv a\pmod n$,则被称$a$为\textbf{模$n$的二次剩余}(quadratic residue modulo $n$)。
    \end{definition}

    例如,当$p=5$时,有:
    \[
        1^{2} \equiv 1,\quad 2^{2} \equiv 4,\quad 3^{2} \equiv 4,\quad 4^{2} \equiv 1,\quad 5^{2} \equiv 0 \pmod 5
    \]
    不难发现,5的二次剩余只有0,1,4。

    又例如,当$p=7$时,有:
    \[
        1^{2} \equiv 1,\quad 2^{2} \equiv 4,\quad 3^{2} \equiv 2,\quad 4^{2} \equiv 2,\quad 5^{2} \equiv 4,\quad 6^{2} \equiv 1,\quad 7^2 \equiv 0\pmod 7
    \]
    即7的二次剩余只有0,1,2,4。

    对于$4k+1$型素数$p$,存在$x$,使得$x^{2}+1 \equiv 0 \pmod p$;而对于$4k+3$型素数$p$,不存在$x$,使得$x^{2}+1 \equiv 0 \pmod p$。该结论将在下节课给出证明。

    利用$4k+3$型素数的性质,我们可以证明$4k+1$型的素数有无穷个。假设$4k+1$型的素数有有限个,记为$p_{1},p_{2},...,p_{n}$,考虑:
    \[N=4p_{1}^{2}p_2^2\cdots p_{n}^2+1\]
    显然,$2\nmid N$且$p_{i} \nmid N,i=1,2,...,n$,所以$N$的因子都是$4k+3$型,即存在$q=4k+3$,使得$N \equiv 0\pmod q$。与二次剩余性质矛盾!所以,$4k+1$型的素数有无穷个。

\end{document}






