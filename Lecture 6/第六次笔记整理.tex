\documentclass[11pt]{article}

\usepackage[a4paper]{geometry}
\geometry{left=2.0cm,right=2.0cm,top=2.5cm,bottom=2.5cm}

\usepackage{ctex}
\usepackage{amsmath,amsfonts,graphicx,amssymb,bm,amsthm,mathrsfs}
\usepackage{algorithm,algorithmicx}
\usepackage[noend]{algpseudocode}
\usepackage{fancyhdr}
\usepackage{hyperref}

\newcounter{counter_exm}\setcounter{counter_exm}{1}
%\newcounter{counter_thm}\setcounter{counter_thm}{1}
%\newcounter{counter_lma}\setcounter{counter_lma}{1}
%\newcounter{counter_dft}\setcounter{counter_dft}{1}
%\newcounter{counter_clm}\setcounter{counter_clm}{1}
%\newcounter{counter_cly}\setcounter{counter_cly}{1}

\newtheorem{theorem}{{\hskip 2em \bf 定理}}
\newtheorem{lemma}[theorem]{\hskip 2em 引理}
\newtheorem{proposition}[theorem]{Proposition}
\newtheorem{claim}[theorem]{\hskip 2em 命题}
\newtheorem{corollary}[theorem]{\hskip 2em 推论}
\newtheorem{definition}[theorem]{\hskip 2em 定义}

\renewcommand{\emph}[1]{\begin{kaishu} #1 \end{kaishu}}

\newenvironment{solution}{{\noindent\hskip 2em \bf 解 \quad}}

\renewenvironment{proof}{{\noindent\hskip 2em \bf 证明 \quad}}{\hfill$\qed$\par}
\newenvironment{example}{{\noindent\hskip 2em \bf 例 \arabic{counter_exm}\quad}}{\addtocounter{counter_exm}{1}\vskip1em\par}

\newenvironment{concept}[1]{{\bf #1\quad} \begin{kaishu}} {\end{kaishu}\par}

\newcommand\E{\mathbb{E}}

\begin{document}

    \pagestyle{fancy}
    \lhead{\kaishu 中国科学院大学}
    \chead{}
    \rhead{\kaishu 2017年秋季学期组合数学}

    \begin{center}
        {\LARGE \bf 组合数学第六讲}\\
    \end{center}
        \begin{kaishu}
            授课时间: 2017年10月17日\quad
            授课教师: 孙晓明
            \hfill 记录人: 孙楚尧\quad 鲍天成
        \end{kaishu}


    \section{思考题}
    考虑二维的随机游走$\{P_t\}_{t\geq 0}$。初始时$P_0=(0, 0)$,每次在上下左右四个方向上等概率选择一个方向移动长度$1$,即递归地定义二维随机向量$P_t$满足:
    \begin{align*}
        &\Pr(P_t=P_{t-1}+(0,1)) = \Pr(P_t=P_{t-1}+(0,-1)) = \\
        &\Pr(P_t=P_{t-1}+(-1,0)) = \Pr(P_t=P_{t-1}+(1,0)) = 1/4
    \end{align*}
    又定义$r_t := \Pr(P_t=(0,0)\land \bigwedge_{k=1}^{t-1} P_{k}\not=(0,0))$,表示游走点在$t$时刻首次回到$(0,0)$的概率。请问:游走点是否一定会回到原点(0,0),即$\sum_{t=1}^\infty r_t=1$是否满足?如果在三维的网格上呢?

    \vskip1em
    \begin{solution}
        先证明如下引理。\footnote{本解法来自李鑫泽同学}
        \begin{lemma}
            设$S_t$表示游走点在$t$时刻的位置,$O$为出发原点,则$\sum_{t=1}^\infty r_t=1$当且仅当满足
            \[\sum \nolimits_{t=1}^\infty \Pr(S_t=O)=\infty\]
        \end{lemma}

        \begin{proof}
            设$I_i$是一个随机变量,其定义为
            \begin{align*}
                 I_i=\begin{cases}
                      1, & \text{if }S_i=O; \\
                      0, & \text{if }S_i \neq O
                 \end{cases}
            \end{align*}
            而随机变量$N$定义为$N=\sum_{i=1}^\infty I_i$。则$N$的期望
            \begin{align*}
                E[N]&=\sum_{i=1}^\infty E[I_i]\\
                    &=\sum_{i=1}^\infty \Pr(S_i=O)\\
                    &=\sum_{k=1}^\infty k\Pr(N=k)\\
                    &=\sum_{k=1}^\infty \Pr(N \geq k)=\sum_{k=1}^\infty\left(\sum_{t=1}^\infty r_t\right)^k
            \end{align*}
            故$\sum_{t=1}^\infty \Pr(S_t=0)=\infty$当且仅当级数$\sum_{k=1}^\infty\left(\sum_{t=1}^\infty r_t\right)^k$不收敛。而$0 \leq \sum \nolimits_{t=1}^\infty r_t \leq 1$,故$\sum \nolimits_{t=1}^\infty r_t=1$当且仅当满足$\sum_{t=1}^\infty \Pr(S_t=0)=\infty$。
        \end{proof}

        \newpage
        有了该引理,我们可以解决上述问题。

        $\textbf{二维随机游走}$:若游走点回到$(0,0)$ 点,设其向上走了$i$步,向下走了$i$步,向左走了$j$步,向右走了$j$步,总共走了$2n$步,故
        \begin{align*}
            \Pr(S_{2n}=O)&=\sum_{i+j=n} 4^{-2n} \frac{2n!}{(i!)^2(j!)^2}\\
            &=\sum_{i=0}^n 4^{-2n}\frac{2n!}{(i!)^2((n-i)!)^2}\\
            &=4^{-2n}{2n \choose n} \sum_{i=0}^n{n \choose i}{n \choose n-i}\\
            &=4^{-2n}{2n \choose n}{2n \choose n}\\
            &=O(n^{-1})
        \end{align*}

        根据\emph{Stirling}公式\footnote {Stirling公式:$n!\sim\sqrt{2\pi n}\;(\frac{n}{e})^n$},上述算式与$\frac{1}{n}$同阶,故$\sum \nolimits_{t=1}^\infty P(S_t=0)=\infty$,由引理可知,$\sum \nolimits_{t=1}^\infty r_t=1$对二维随机游走成立,即二维随机游走是常返的。\par

        \vskip1em
        $\textbf{三维随机游走}$:三维随机游走时,同理可得
        \begin{align*}
            \Pr(S_{2n}=0)&=\sum_{i+j\leq n} 6^{-2n} \frac{2n!}{(i!)^2(j!)^2[(n-i-j)!]^2}\\
            &=6^{-2n}{2n \choose n} \sum_{i+j\leq n} \frac{(n!)^2}{(i!)^2(j!)^2[(n-i-j)!]^2}\\
            &=6^{-2n}{2n \choose n} \sum_{i+j+k=n} {n \choose i,j,n-i-j}^2\\
            &\leq 6^{-2n}{2n \choose n} \left(\max_{i+j+k=n} {n \choose i,j,n-i-j}\right) \left(\sum_{i+j+k=n} {n \choose i,j,n-i-j}\right)\\
            &= 6^{-2n}{2n \choose n} 3^n{n\choose \lfloor n/3\rfloor,\lfloor n/3\rfloor,n-2\lfloor n/3\rfloor}\\
            &=O(n^{-1.5})
        \end{align*}
        故$\sum_{t=1}^\infty P(S_t=0)$收敛,由引理可知,$\sum \nolimits_{t=1}^\infty r_t=1$对三维随机游走不成立。
    \end{solution}

    \section{指数生成函数}

    \subsection{$n$位数中数字出现奇数次或偶数次}
    \begin{example}
        \emph{确定满足下面条件的$n$位数的个数$h_n$: 每个数字都为1,2,3中的一个且数字1出现偶数次,数字2出现奇数次。}
    \end{example}

    \begin{solution}
        若按照之前的递推方法,我们将需要定义4个递推式:$A_{00}$,$A_{01}$,$A_{10}$,$A_{11}$来表示数字1和数字2个数的奇偶性。现考虑使用一种更简便的方法,用指数生成函数来解决问题。

        设数字1出现$i$次,数字2出现$j$次,数字3出现$n-i-j$次,则由1,2,3组成的$n$位数共有$\displaystyle \frac{n!}{i!j!(n-i-j)!}$ 个。即要求
        \[
            h_n=\sum_{\substack{i+j\le n \\ 2\mid i, 2\nmid j}} \frac{n!}{i!j!(n-i-j)!}
        \]
        于是
        \[
            \frac{h_n}{n!}=\sum_{\substack{i+j\le n \\ 2\mid i, 2\nmid j}} \frac{1}{i!j!(n-i-j)!}
        \]
        其形式类似于卷积。于是我们定义生成函数
        \[
            H(x):=\sum_{\substack{n\ge 0}} \frac{h_n}{n!}x^n
        \]
        取$h_n= 1$,可得
        \begin{align}
            1+\frac{x}{1!}+\frac{x^2}{2!}+\frac{x^3}{3!}+\dots =e^x
        \end{align}

        我们发现,对$(1)^3$式,$x^n$项的系数为$\displaystyle \sum_{\substack{i+j\le n}} \frac{1}{i!j!(n-i-j)!}$。若要使得$i$,$j$满足$2\mid i, 2\nmid j$, 构造级数
        \begin{align}
            1+\frac{x^2}{2!}+\frac{x^4}{4!}+\dots =\frac{e^x+e^{-x}}{2} \\
            \frac{x}{1!}+\frac{x^3}{3!}+\frac{x^5}{5!}+\dots =\frac{e^x-e^{-x}}{2}
        \end{align}
        其中$(2)$式中仅有偶数次项,$(3)$式中仅有奇数次项,因此对于$(1)\cdot (2)\cdot (3)$,$x^n$项系数为
        \[
            \sum_{\substack{i+j\le n \\ 2\mid i, 2\nmid j}} \frac{1}{i!j!(n-i-j)!} = h_n
        \]
        则
        \begin{align*}
            H(x) ={} & \frac{e^x+e^{-x}}{2}\cdot \frac{e^x-e^{-x}}{2}\cdot e^x \\
                 ={} & \frac{1}{4}[e^{2x}-e^{-2x}]e^x \\
                 ={} & \frac{1}{4}[e^{3x}-e^{-x}] \\
                 ={} & \frac{1}{4}\sum_{\substack{n\ge 0}} \frac{{(3x)}^n}{n!} - \frac{1}{4}\sum_{\substack{n\ge 0}} \frac{{(-x)}^n}{n!}
        \end{align*}
        求得
        \[
            h_n=\frac{3^n-{(-1)}^n}{4}.
        \]

        由此我们可以引出这个概念的具体定义:

        \begin{concept}{指数生成函数}
            对数列$h_0, h_1, h_2, \dots , h_n, \dots $ 利用单项式的集合$\lbrace 1, x, \frac{x^2}{2!}, \dots , \frac{x^n}{n!}, \dots \rbrace$得到生成函数。这些单项式出现在泰勒级数
            \[
                e^x=\sum_{n=0}^{\infty } \frac{x^n}{n!}=1+x+\frac{x^2}{2!}+\dots +\frac{x^n}{n!}+\dots
            \]
            中。关于该单项式集合的生成函数称为\textbf{指数生成函数},定义为
            \[
                g^{(e)}(x)=\sum_{n=0}^{\infty } h_n\frac{x^n}{n!}=h_0+h_1x+h_2\frac{x^2}{2!}+\dots +h_n\frac{x^n}{n!}+\dots
            \]
        \end{concept}

    \end{solution}

    \subsection{$n$位数中数字出现偶数次及3的倍数次}
    \begin{example}
        \emph{确定满足下面条件的$n$位数的个数$h_n$:每个数字都为1,2,3中的一个且数字1出现偶数次,数字2出现3的倍数次。}
    \end{example}

    \begin{solution}
        对于这道题目,若要使用递推方法将极其复杂。于是我们定义指数生成函数
        \[
            H(x):=\sum_{\substack{n\ge 0}} \frac{h_n}{n!}x^n
        \]
        与\textbf{2.1}同理,我们分别只需要$(1)$式中的偶数次项和3的倍数项。构造级数
        \begin{gather}
            1+\frac{x^2}{2!}+\frac{x^4}{4!}+\dots =\frac{e^x+e^{-x}}{2} \\
            1+\frac{x^3}{3!}+\frac{x^6}{6!}+\dots =\frac{e^x+e^{\omega x}+e^{{\omega }^2}x}{3}
        \end{gather}
        则生成函数
        \begin{align*}
            H(x) =(1)\cdot (4)\cdot (5) ={} & e^x\cdot \frac{e^x+e^{-x}}{2}\cdot \frac{e^x+e^{\omega x}+e^{{\omega }^2}x}{3} \\
                 ={} & \frac{1}{6}[(e^x+e^{-x})(e^x+e^{\omega x}+e^{{\omega }^2}x)e^x] \\
                 ={} & \frac{1}{6}[e^{3x}+e^{(2+\omega )x}+e^{(2+{\omega }^2)x}+e^x+x^{\omega x}+e^{{{\omega }^2}x}]
        \end{align*}
        得到
        \[
            h_n=\frac{1}{6}[3^n+{(2+\omega )}^n+{(2+{\omega }^2)}^n+1+{\omega }^n+{\omega }^{2n}].
        \]
    \end{solution}

    \section{容斥原则(Inclusive-Exclusive Principle)}

    \subsection{容斥原则及证明}
    \begin{concept}{容斥原则}
       设$A_1$,$A_2$,\dots,$A_n$是全集$S$的$n$个子集,则
       \[
       |A_1 \cup A_2 \cup \dots \cup A_n|=\sum \limits _{i=1}^n|A_i|-\sum_{i \le j}|A_i \cap A_j|+ \dots + (-1)^{n-1}|A_1 \cap A_2 \cap A_3 \cap \dots \cap A_n|
         \]
    \end{concept}
    \begin{proof}{(数学归纳法)}
        当$n=2$时,等式容易证明是成立的。假设$n=s$ $(s \geq 2,s \in \mathbb N^+ )$时结论成立,则当$n=s+1$时,
        \begin{align*}
           |\bigcup \limits_{i=1}^{s+1}A_i|&=|(\bigcup \limits_{i=1}^s A_i)\cup A_{s+1}|\\
           &=|\bigcup \limits_{i=1}^{s}A_i|+|A_{s+1}|-|(\bigcup \limits_{i=1}^s A_i)\cap A_{s+1}|\\
           &=|\bigcup \limits_{i=1}^{s}A_i|+|A_{s+1}|-|\bigcup \limits_{i=1}^s (A_i \cap A_{s+1})|\\
        \end{align*}
        对等式中的第一项和第三项分别使用$n=s$时的假设并进行化简,可以得到
        \[
            |\bigcup_{i=1}^{s+1}A_i|=\sum_{k=1}^s(-1)^{k-1}\sum_{1\leq i_1<\ldots< i_k\leq s}|A_{i_1}\cap \ldots \cap A_{i_k}|+|A_{s+1}|-
            \sum_{k=1}^s(-1)^{k-1}\sum_{1\leq i_1<\ldots< i_k\leq s}|A_{i_1}\cap \ldots \cap A_{i_k}\cap A_{s+1}|
        \]
        合并上式即为
        \[
            |\bigcup \limits_{i=1}^{s+1}A_i|= \sum \limits _{k=1}^{s+1}(-1)^{k-1} \sum_{1\leq i_1 < i_2 \dots < i_{k}\leq s+1}|A_{i_1} \cap A_{i_2} \cap \dots \cap A_{i_k}|
        \]
        综上,由数学归纳法,容斥原则成立。
    \end{proof}

    但事实上,我们也可以用组合的方法证明容斥原则,证明如下:

    \begin{proof}{(组合方法)}
        对于集合$A=A_1 \cup A_2 \cup \dots \cup A_n$中的任意一个元素$x$,考虑其在两边的计数次数。假设$x$在集合$A_{i_1},A_{i_2} \dots A_{i_k}$中出现,而不在$A_{i_j}$出现$(j \neq 1,2,\dots k)$。 在等式左侧$x$只被计数一次。而在等式右侧,在$\sum \nolimits _{i=1}^n|A_i|$中,$x$被计数$k \choose 1$次; 在$-\sum_{i\le j}|A_i \cap A_j|$中,$x$被计数$- {k \choose 2}$ 次。以此类推,$x$在右侧被计数的次数为
        \[
            {k \choose 1} - {k \choose 2} + {k \choose 3} + \dots + (-1)^{k-1} {k \choose k}
        \]
        根据二项式定理
        \[
            (1-1)^k={k \choose 0} - {k \choose 1} + {k \choose 2} - {k \choose 3} - \dots + (-1)^{k-1} {k \choose k}
        \]
        可知$x$在等式右侧被计数$1$次,即$A$中的任意元素在等式两侧被记录了相同的次数,原命题得证。
    \end{proof}

    容斥原理也经常被写成如下形式:
    \[
        |\overline{A_1} \cap \overline{A_2} \cap \dots \cap \overline{A_n}|=|S|-\sum_{i=1}^n|A_i|+\sum_{i \le j}|A_i \cap A_j|- \dots + (-1)^{n}|A_1 \cap A_2 \cap A_3 \cap \dots \cap A_n|
    \]
    其中$S$表示全集,这种形式在求解某一些问题时可能会比原来形式更加好用,详见以下例子。

    \subsection{容斥原则的应用}

    \begin{example}
        \emph{求小于$100$的素数的个数。}
    \end{example}

    \begin{solution}
    容易得到,小于$100$的合数必然含因子$2$,$3$,$5$或$7$ $(11^2=121>100)$。故此时可以通过容斥原理求解,其中$A_2$表示$100$以内被$2$整除的数的集合。
     \par 此时以$\pi(100)$表示小于$100$的素数个数,即
     \begin{align*}
         \pi(100)&=|S|-|\overline{A_2} \cap \overline{A_3} \cap \overline{A_5} \cap \overline{A_7}|-1+4\text{(除去1并补充2,3,5,7 这4个数)}\\
         &=100-\left[\frac{100}{2}\right]-\left[\frac{100}{3}\right]-\left[\frac{100}{5}\right]-\left[\frac{100}{7}\right]+\left[\frac{100}{2 \times 3}\right]\\
         &+\left[\frac{100}{2 \times 5}\right]+\left[\frac{100}{2 \times 7}\right]+\left[\frac{100}{3 \times 5}\right]+\left[\frac{100}{3 \times 7}\right]+\left[\frac{100}{5 \times 7}\right]-\left[\frac{100}{2 \times 3\times 5}\right]\\
         &-\left[\frac{100}{2 \times 3\times 7}\right]-\left[\frac{100}{3 \times 5\times 7}\right]-1+4 \quad \text{(其余项取整后为0)}\\
         &=100-50-33-20-14+16+10+7+6+4+2-3-2-1-1+4\\
         &=25.
        \end{align*}
    \end{solution}

    \begin{example}
        求不定方程解的个数
    \end{example}

    \textbf{情况一:} $n$元不定方程
    \[
        \lbrace
            (x_1, x_2, x_3, \dots , x_n)\mid
            x_1+x_2+x_3+\dots +x_n=m, x_i\in \mathbb{N},
            x_1\ge 2, x_2\ge 7, x_3\ge 5
        \rbrace
    \]

    \begin{solution}
        我们定义$z_1:=x_1-2;z_2:=x_2-7;z_3:=x_3-5 $,即原不定方程转化为
        \[
        \lbrace
            (z_1, z_2, z_3, x_4, \dots , x_n)\mid
            z_1+z_2+z_3+x_4+\dots +x_n=m-14, x_i, z_i\in \mathbb{N}
        \rbrace
        \]
        易知共有$\displaystyle {{m-15+n}\choose {n-1}}$个解。
    \end{solution}

    \vskip1em
    \textbf{情况二:} $n$元不定方程
    \[
        \lbrace
            (x_1, x_2, x_3, \dots , x_n)\mid
            x_1+x_2+x_3+\dots +x_n=m, x_i\in \mathbb{N},
            x_1\le 3, x_2\le 7, x_3\le 5
        \rbrace
    \]

    \begin{solution}
        其中解的可行范围有所变化。大于等于的条件对我们来说很容易解决,要考虑小于等于的条件,我们可以先引入$A_1$为条件是$x_1\ge 4$的$n$元不定方程的解集,$A_2$为条件是$x_2\ge 8$的$n$元不定方程的解集,$A_3$为条件是$x_3\ge 6$的$n$元不定方程的解集。

        这样我们就将条件中的小于等于转化成大于等于,很容易求得$A_1,A_2,A_3$解的个数。如情况一做法一致,
        再定义$z_1:=x_1-4;z_2:=x_2-8;z_3:=x_3-6$。同理,可以求出不定方程解的个数为
        \begin{align*}
            |\overline{A_1} \cap \overline{A_2} \cap \overline{A_3}|
            = {} & |S|-|A_1|-|A_2|-|A_3|+|A_1 \cap A_2|+|A_1 \cap A_3|+|A_2 \cap A_3|-|A_1 \cap A_2 \cap A_3| \\
            = {} & {{m+n-1}\choose {n-1}}-{{m+n-5}\choose {n-1}}-{{m+n-9}\choose {n-1}}-{{m+n-7}\choose {n-1}}+ \\
              {} & {{m+n-13}\choose {n-1}}+{{m+n-11}\choose {n-1}}+{{m+n-15}\choose {n-1}}-{{m+n-19}\choose {n-1}}
        \end{align*}

    \end{solution}

    \subsection{错位排列}

    \begin{example}
        \emph{将$n$封不同的信随机装进$n$个信封,求每封信都装错的概率。}
    \end{example}

    \begin{solution}
        考虑所求情境:即1号信封中装的不是1号信,2号信封中不是2号信,$\dots $,$n$号信封中不是$n$号信。用置换表示,则所求情况可以表示为:
        \[
            D=\lbrace
                \pi \in S_n \arrowvert \pi(1)\ne 1, \pi(2)\ne 2, \dots , \pi(n)\ne n
              \rbrace
        \]
        其中$S_n$为1至$n$的置换全集。现定义1号信封内所装信件正好为1号信的排列集合为$A_1$,2号信封内装2号信的排列集合为$A_2$,以此类推,$n$号信封内装$n$号信的排列集合为$A_n$。易知$|A_i|=(n-1)!,|A_{i_1}\cap A_{i_2}|=(n-2)!\dots ,|A_{i_1}\cap A_{i_2}\cap \dots \cap A_{i_k}|=(n-k)!$,则
        \begin{align*}
            |D| = {} & |\overline{A_1} \cap \overline{A_2} \cap \dots \cap \overline{A_n}| \\
                = {} & n!-{n \choose 1}(n-1)!+{n \choose 2}(n-2)!-\dots +(-1)^n{n \choose n}(n-n)! \\
                = {} & \sum_{i=0}^n (-1)^i{n \choose i}(n-i)! \\
                = {} & n!\sum_{i=0}^n \frac{(-1)^i}{i!} \\
                \sim {} & n!\cdot e^{-1}
        \end{align*}
        由此得到错位排列数目$D_n$公式:

        \begin{theorem}
            对于$n\ge 1$,错位排列的个数为
            \[
                D_n=n!(1-\frac{1}{1!}+\frac{1}{2!}-\frac{1}{3!}+\dots +(-1)^n\frac{1}{n!})
            \]
            从任意排列中取到错位排序的概率随着$n$的增大趋近于$e^{-1}$。
        \end{theorem}
    \end{solution}

    \begin{example}
        \emph{现有$n$对夫妻要坐在一个圆桌旁,求任意一对夫妻都不相邻的圆排列数量。}
    \end{example}

    \begin{solution}
        我们先考虑第一对夫妻坐在相邻位置的情况,此时可以假设有“胶水”将这一对夫妻粘在一起,然后进行一个有$2n-1$个元素的圆排列,但这对夫妻可以选择两个人的落座方式,因此还需乘上$2$。故第一对夫妻坐在相邻位置的圆排列数是$2\times (2n-2)!$。设$A_i$表示第$i$对夫妻坐在相邻位置的圆排列形成的集合。故原题即求$|\overline{A_1} \cap \overline{A_2} \cap \dots \cap \overline{A_n}|$,根据容斥原理及上面对一对夫妻的分析,可以得到符合题意的圆排列数量为$\sum \nolimits_{k=0}^n (-1)^{n+1}\times 2^k{n \choose k}(2n-1-k)!$,概率约为$e^{-1}$。
    \end{solution}

    装错信封问题还有很多类似的例子,其思路都是类似的,利用容斥公式的第二种形式可以帮助我们解决问题。不过还有一种更加复杂的装错信封问题,十分有趣。

    \subsection{100囚徒问题}
    \begin{example}
        \emph{有$80$个学生,编号$1\sim80$。在一个封闭的房间里放置$80$个信封,编号也为$1\sim80$。另外有编号为$1\sim80$的$80$封信被随机放在信封中,每个信封内放一张。学生依次进入房间,每人最多打开$40$个信封,如果打开信封后里面的信的编号与自己的编号一致,则该学生成功达成目标。当且仅当所有学生达成目标,学生获胜。学生在进入房间前可以商讨一个策略。使用怎样的策略才能有较大的可能获胜?}
     \end{example}

    \begin{solution}
    由于时间不足,在课上我们只提出了策略,具体获胜概率计算将留到下节课进行。下简述该策略:
    \begin{itemize}
        \item [1.] 每个学生首先打开与自己编号相同的信封;
        \item [2.] 如果信封里就是与其编号相同的信,则达成目标;
        \item [3.] 否则,他接下来打开与该信封内信相同编号的信封;
        \item [4.] 重复2,3步,直至达成目标或打开40 个信封为止。
    \end{itemize}
    可以看出,此方法获胜的概率即为从置换群$S_{80}$中抽出一个没有元素数大于$40$的循环的置换概率。
    \par 事实上,这就是\emph{100囚徒问题(100 prisoners problem)},由丹麦计算机科学家Peter Bro Miltersen于2003年首次提出。
    \end{solution}















\end{document}





